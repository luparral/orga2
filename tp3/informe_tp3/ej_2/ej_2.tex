\section{Ejericio 2}

\subsection{Creación de las estructuras de la IDT} 

En la función \texttt{idt\_inicializar} inicializamos las entradas iniciales de la IDT (de la 20 a la 31 estan reservadas para Intel).
Ademas inicializamos las entradas 32 para reloj, 33 para teclado y 70 para las syscalls a ser utilizadas por las tareas.
Las entradas de la IDT estan definidas por una macro en la que elegimos el número que representa su índice. 
A la entrada para la syscall le damos los siguientes atributos: 0xEE00.
A todas las demás: 0x8E00.


Estas difieren ya que la syscall al ser llamada por la tarea, necesita tener dpl = 3 para que esta pueda accederla, mientras que las otras interrupciones no.

Luego para que el procesador utilice la IDT utilizamos la interrupción lidt y configuramos el controlador de interrupciones con las funciones \texttt{resetear\_pic} y \texttt{habilitar\_pic}, provistas por la cátedra.