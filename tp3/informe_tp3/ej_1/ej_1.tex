\section{Ejericio 1}

\subsection{Completar la tabla de descriptores globales}
Para esto completamos el indice 0 de la GDT con el descriptor nulo. Las siguientes 7 entradas por requerimientos de este trabajo debian ser nulas.
A continuación, completamos con entradas para código y datos para sistema operativo (nivel 0) y para usuario (nivel 3). Los segmentos de código y datos del sistema operativo están descritos en las entradas 8 y 9 respectivamente, mientras que los de usuario en las entradas 10 y 11. 
El campo límite para estas entradas es 0xF3FF y la base 0x0000, con granularidad 1, ya que direccionan  500MB de memoria. El tipo para los los descriptores de código es 0x8, correspondiente a código de sólo-ejecución, mientras que para los de datos es de 0x2, para segmentos de datos de lectura/escritura. También es relevante indicar que para todas estas entradas, el campo p es 1, ya que los segmentos se encuentran presentes.
También tenemos una entrada de datos para el segmento de video, con base 0xB800 y límite 0xC0000, con dpl 0, pues es para ser utilizado sólo por el kernel, presente y permisos de lectura/escritura. Este descriptor tendrá índice 12. 

\subsection{Pasaje a modo protegido}
En primer lugar, cambiamos el modo de video a 80x50.
Habilitamos a20 llamando a \texttt{habilitar\_A20}.
Cargamos la GDT  con la instrucción \texttt{lgdt}.

Para pasar a modo protegido, seteamos en 1 el bit PE del registro CR0 y hacemos jump far con un selector de segmento de código de nivel 0 y offset \texttt{modo\_protegido} (etiqueta que declaramos a continuación)

Una vez en modo protegido, asignamos los selectores de segmentos. Como ds, ss elegimos segmento de datos de nivel 0, y para fs el segmento de video.

Seteamos la pila del kernel en la direccion 0x27000.

\subsection{Inicialización de la pantalla y juego}

Usamos las funciones game\texttt{\_inicializar}, \texttt{screen\_inicializar} y \texttt{game\_inicializar\_botines}.

La función \texttt{screen\_inicializar}, que pinta el fondo de gris, y el recuadro para cada jugador en azul y rojo.
\texttt{game\_inicializar} arranca todas las variables del juego para su uso.
\texttt{game\_inicializar\_botines} dibuja todos los botines en la pantalla
