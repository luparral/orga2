\documentclass[10pt, a4paper,english,spanish]{article}
%\documentclass[10pt,a4paper]{article}
\usepackage[utf8]{inputenc} % para poder usar tildes en archivos UTF-8
\usepackage[spanish]{babel} % para que comandos como \today den el resultado en castellano
\usepackage{fullpage} %small margins
\usepackage[parfill]{parskip} %genera saltos entre parrafos
\usepackage{color}
\definecolor{gray}{gray}{0.35}
\usepackage{listings}
\usepackage{enumitem}
\usepackage{amsmath} %big brackets
\usepackage[pdftex]{graphicx}
\lstset{
    numbers=left,
    breaklines=true,
    tabsize=2,
    basicstyle=\ttfamily\color{gray},
}
\setlength{\parindent}{8pt}
\usepackage{mathtools}
\usepackage[margin=50pt]{geometry}
\usepackage{amsfonts}
\usepackage{flafter}
\usepackage{multicol}
\usepackage{caption}
\usepackage{subcaption}

\begin{document}

\section{Blur}
\subsection{Introducción}
Blur es un filtro que suaviza y distorsiona la imagen, dependiendo de cuanta proporción se le aplique. En este caso los algoritmos presentados y desarrollados son de un suavizado leve que es una de las instancias del Blur Gaussiano; consiste en tomar un pixel de la imagen y sus 8 mas cercanos, calcular su promedio con cada canal de color (Red Green Blue) y asignarlo al pixel anteriormente mencionado, este proceso se repite para cada pixel de la imagen. 

\newpage
\subsection{Desarrollo Implementación C}

\newpage
\subsection{Desarrollo Implementación ASM1}

\newpage
\subsection{Desarrollo Implementación ASM2}

\newpage
\subsection{Experimentación}

\newpage
\subsection{Conclusion}

\end{document}