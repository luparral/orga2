\section{Blur}
\subsection{Introducción}
\textit{Blur} es un filtro que suaviza una imagen. Le asigna a cada pixel el promedio con sus pixeles vecinos.\\ Es decir:

\begin{verbatim}
  m[j][i][k] = (m[j-1][i-1][k] + m[j-1][i][k] + m[j-1][i+1][k] + 
              m[j] [i-1][k] + m[j] [i][k] + m[j] [i+1][k] + 
              m[j+1][i-1][k] + m[j+1][i][k] + m[j+1][i+1][k] ) / 9
\end{verbatim}

De esta forma, la salida es la imagen en donde para cada pixel, se realizó el "suavizado" según el cálculo anterior.

\subsection{Pseudocódigo en C}

El pseudocódigo para una iteración del ciclo de la implementación de C provista por la cátedra es:

\begin{lstlisting}

//TODO: Mejorar este pseudocodigo

for(iw=0;iw<(int)w;iw++) {
    for(ii=0;ii<4;ii++) {
      m_row_1[iw][ii] = m[0][iw][ii];
    }
  }
  for(ih=1;ih<(int)h-1;ih++) {
    m_tmp = m_row_0;
    m_row_0 = m_row_1;
    m_row_1 = m_tmp;
    for(iw=0;iw<(int)w;iw++) {
      for(ii=0;ii<4;ii++) {
        m_row_1[iw][ii] = m[ih][iw][ii];
      }
    }
    for(iw=1;iw<(int)w-1;iw++) {
      for(ii=0;ii<4;ii++) {
        m[ih][iw][ii] = ( 
          (int)m_row_0[iw-1][ii] + (int)m_row_0[iw][ii] + (int)m_row_0[iw+1][ii] +
          (int)m_row_1[iw-1][ii] + (int)m_row_1[iw][ii] + (int)m_row_1[iw+1][ii] +
          (int)m[ih+1][iw-1][ii] + (int)m[ih+1][iw][ii] + (int)m[ih+1][iw+1][ii] ) / 9;
      }
    }
  }

\end{lstlisting}

\subsection{Implementación 1 en ASM}
La primera implementación consiste en procesar de a un pixel por iteración.\\

La implementación comienza pidiendo memoria a través de \textit{malloc} para guardar dos filas de la imagen a procesar.\\Para la primera vez, se cargan las dos primeras filas, copiándolas en los dos espacios de memoria recien pedidos.\\
Luego, se ejecuta el ciclo del filtro, levantando de a 4 píxeles de tres lugares distintos, los cuales son:
\begin{enumerate}
\item la imágen a blurear
\item cada una de las dos filas copiadas en los espacios de memoria pedidos inicialmente
\end{enumerate}
A pesar de que se levantan de a cuatro píxeles, solo usamos 3 de ellos, ya que necesitamos 3 de cada uno para conseguir el promedio dado por el pseudocódigo provisto en el enunciado del TP.\\
Una cuestión importante es que los 4 píxeles que levantamos de la imagen original son tomados "defasados" 1 pixel a la izquierda, con el fin de no "salirnos" e la imagen en el cálculo del último pixel. Notar que leantar defasados a la izuqierda un pixel no hace que no accedamos a memoria que no es nuestra, ya que el pixel "fuera" de la imagen que estamos pidiendo en la primer iteración, en realidad no está fuera de la imagen sino que es de la fila anterior (copiada en uno de los dos espacios de memoria pedidos inicialmente).\\
A continuación, shifteamos 1 pixel a derecha para subsanar lo recientemente explicado y tener los 3 píxeles que utilizaremos en la parte menos significativa.\\
Una vez con los 9 píxeles guardados en registros XMM, calculamos el promedio y lo aplicamos al pixel correspondiente. 
Finalmente, avanzamos un pixel y volvemos a iterar.\\

\subsection{Implementación 2 en ASM}
Esta implementación propone procesar de a cuatro pixeles.
El siguiente es un pseudocódigo para una iteración del ciclo de esta implementación:

\begin{lstlisting}
//TODO: pseudocodigo :P
\end{lstlisting}


\subsection{Resultados}
A continuación, detallamos los resultados obtenidos a través de la experimentación con las distintas implementaciones en C y Assembler para este filtro y las conclusiones a las que llegamos tras el estudio de los mismos.\\
Las siguientes son las tablas resultantes de tomar el promedio de todos los tiempos medidos para imágenes tendientes a un mismo color, para los distintos tamaños evaluados, según se detalla en la introducción.\\

\begin{tabular}{| l | l | l | l | l |}
\hline
Implementación & Color & 1600 píxeles & 90000 píxeles & 360000 píxeles\\
\hline
Blur ASM1 & azul & 29333.87 & 1389083.47&  5590953.07\\ 
\hline
Blur ASM1 & blanco & 28953.00&  1319802.00 & 5539200.67\\ 
\hline
Blur ASM1 & mixto & 28765.56  &1369988.44  & 5712646.67\\ 
\hline
Blur ASM1 & negro & 28342.25  &1367054.50  & 5607272.25\\
\hline
Blur ASM1 & rojo & 28703.00  &1370567.00  & 5651159.63\\
\hline
Blur ASM1 & verde & 28895.03  &1363316.78  & 5635944.92\\ 
\hline
Promedio & &  28895.03  &1363316.78 & 5635944.92\\
\hline
Desvio estándard  && 373.53  &23102.01 &  69973.61\\
\hline
Porcentaje de desviación  && 1.29\%& 1.69\%& 1.24\%\\
\hline
\end{tabular}

\begin{tabular}{| l | l | l | l | l |}
\hline
Implementación & Color & 1600 píxeles & 90000 píxeles & 360000 píxeles\\
\hline
Blur ASM2 & azul & 17992.73 &  750119.40&  3141036.87\\ 
\hline
Blur ASM2 & blanco & 18219.67  & 770902.67 & 3104900.00\\ 
\hline
Blur ASM2 & mixto &  17272 & 752688.88 & 3157686.11\\ 
\hline
Blur ASM2 & negro & 18470.50  & 776735.75 & 3127282.25\\
\hline
Blur ASM2 & rojo & 17351.25 &  732310.25 & 3165519.38\\
\hline
Blur ASM2 & verde & 17755.25 & 765754.75 & 3170724.75\\ 
\hline
Promedio & &  17843.57  & 758085.28 & 3144524.89\\
\hline
Desvio estándard  && 476.15  & 16296.47  & 25219.14\\
\hline
Porcentaje de desviación  && 2.67\% & 2.15\% & 0.80\%\\
\hline
\end{tabular}

\begin{tabular}{| l | l | l | l | l |}
\hline
Implementación & Color & 1600 píxeles & 90000 píxeles & 360000 píxeles\\
\hline
Blur C & azul & 306557.40 & 14795203.80 & 59272448.00\\ 
\hline
Blur C & blanco & 272834.33 & 15306523.00 & 57771410.67\\ 
\hline
Blur C & mixto &  297517.11 & 14445272.00 & 57421879.56\\ 
\hline
Blur C & negro & 331603.75 & 16997981.00 & 59230171.00\\
\hline
Blur C & rojo & 286341.13 & 14904426.13 & 57154062.00\\
\hline
Blur C & verde & 287608.75 & 14182027.25 & 58559933.00\\ 
\hline
Promedio & &  297077.08 & 15105238.86 & 58234984.04\\
\hline
Desvio estándard  && 20370.47  & 1004720.01 & 918341.11\\
\hline
Porcentaje de desviación  && 6.86\% & 6.65\% & 1.58\%\\
\hline
\end{tabular}

Finalmente, tomando los promedios para todos los tipos de imágenes para cada implementación y tamaño, logramos el siguiente gráfico que permite evaluar comparativamente la performance de las distintas implementaciones para los distintos tamaños.

\begin{tabular}{| l | l | l | l|}
\hline
Implementación  & 1600 píxeles & 90000 píxeles & 360000 píxeles\\
\hline
Blur ASM1  & 17843.57 & 758085.28 & 2618274.34\\
\hline
Blur ASM2  & 28895.03 & 1363316.78 & 5635944.92\\
\hline
Blur C & 297077.08 & 15105238.86 & 58234984.04\\
\hline
\end{tabular}

\begin{figure}[ht]
\centering
\includegraphics[width=90mm]{blur/graficoBlur.png}
%\caption{A simple caption \label{overflow}}
\end{figure}

Para evaluar si había diferencias en imágenes de un color constante, realizamos los mismos experimentos en este set de caso de pruebas, compuesto con imágenes puramente verdes, azules, rojas, blancas y negras. A continuación, las tablas con resultados de estos experimentos, junto con el gráfico resultante.\\

\begin{tabular}{| l | l | l | l | l |}
\hline
Implementación & Color & 1600 píxeles & 90000 píxeles & 360000 píxeles\\
\hline
Blur ASM1 & Uniforme Blanca & 25158.00 & 1415168.00 & 5707722.00\\ 
\hline
Blur ASM1 & Uniforme Negra & 25008.00 & 1417360.00 & 5821901.00\\ 
\hline
Blur ASM1 & Uniforme Roja & 23542.00 & 1429305.00 & 5522346.00\\ 
\hline
Blur ASM1 & Uniforme Verde & 24259.00 & 1424455.00 & 5663435.00\\
\hline
Blur ASM1 & Uniforme Azul & 23421.00 & 1384261.00 & 5752064.00\\
\hline
Promedio & &  24277.60 & 1414109.80 & 5693493.60\\
\hline
Desvio estándard  &&  803.71 & 17610.73 & 112156.61\\
\hline
Porcentaje de desviación  &&  3.31\% & 1.25\% & 1.97\%\\
\hline
\end{tabular}


\begin{tabular}{| l | l | l | l | l |}
\hline
Implementación & Color & 1600 píxeles & 90000 píxeles & 360000 píxeles\\
\hline
Blur ASM2 & Uniforme Blanca & 13297.00 & 769413.00 & 3180362.00\\ 
\hline
Blur ASM2 & Uniforme Negra & 15364.00 & 777099.00 & 3055456.00\\ 
\hline
Blur ASM2 & Uniforme Roja & 14031.00 & 776847.00 & 3132449.00\\ 
\hline
Blur ASM2 & Uniforme Verde & 16060.00 & 743686.00 & 3191171.00\\
\hline
Blur ASM2 & Uniforme Azul & 14278.00 & 778483.00 & 3159916.00\\
\hline
Promedio & &   14606.00 & 769105.60 & 3143870.80\\
\hline
Desvio estándard  &&  1100.04 & 14645.90 & 54254.03\\
\hline
Porcentaje de desviación  &&   7.53\% & 1.90\% & 1.73\%\\
\hline
\end{tabular}

\begin{tabular}{| l | l | l | l | l |}
\hline
Implementación & Color & 1600 píxeles & 90000 píxeles & 360000 píxeles\\
\hline
Blur C & Uniforme Blanca & 306566.00 & 14236027.00 & 56984192.00\\ 
\hline
Blur C & Uniforme Negra & 303997.00 & 14469139.00 & 56284032.00\\ 
\hline
Blur C & Uniforme Roja & 300376.00 & 14622681.00 & 55996348.00\\ 
\hline
Blur C & Uniforme Verde & 295486.00 & 14840694.00 & 56661072.00\\
\hline
Blur C & Uniforme Azul & 300286.00 & 14917069.00 & 55887204.00\\
\hline
Promedio & &   301342.20 & 14617122.00 & 56362569.60\\
\hline
Desvio estándard  &&  4203.58 & 277090.20 & 458742.02\\
\hline
Porcentaje de desviación  &&   1.39\% & 1.90\% & 0.81\%\\
\hline
\end{tabular}

A partir de esta información, logramos la siguiente tabla que recopila los promedios y nos permite realizar un gráfico comparativo de implementaciones según tamaño.\\



\begin{tabular}{| l | l | l | l|}
\hline
Implementación  & 1600 píxeles & 90000 píxeles & 360000 píxeles\\
\hline
Blur ASM1  & 24277.6 & 1414109.8 & 5693493.6\\
\hline
Blur ASM2  &  14606 & 769105.6 & 3143870.8\\
\hline
Blur C & 301342.2 & 14617122 & 56362569.6\\
\hline
\end{tabular}

\begin{figure}[ht]
\centering
\includegraphics[width=90mm]{blur/graficoBlur_cte.png}
%\caption{A simple caption \label{overflow}}
\end{figure}

\subsubsection{Algunas conclusiones}
Dentro de cada implementación, vemos que el tiempo de ejecución aumenta a medida que aumenta el tamaño de las imágenes de prueba, lo cual era en cierto sentido lo que nos indicaba nuestro intuición que iba a suceder.\\
Así mismo, la implementación más performante resultó ser ASM2.\\
Este resultado no pareciera depender de la cantidad de instrucciones, dado que ambas implementaciones de Assembler poseen una cantidad muy similar de las mismas. 
%todo: me parece que tiene que ver con los accesos a memoria, chequear.
También pudimos corroborar, que para este filtro, el tipo de imagen elegido no afectaba en la performance de la implementación, ya que los porcentajes de desvío son muy pequeños, lo que indica que los promedios de las mediciones de los distintos tipos no se alejan significativamente de la media.\\


